\documentclass{article}
\usepackage[utf8]{inputenc}
\usepackage{hyperref}

\title{Findtastic}
\author{
    Daniel Polanco Pérez \\
    \texttt{regulardani1@gmail.com} \\
    \and
    María Fernanda Suárez González \\
    \texttt{mariafernandasuarezglez@gmail.com} \\
    \and
    Adrián Navarro Foya \\
    \texttt{adriannf0108@gmail.com}
}

\begin{document}

\maketitle

\section*{Introducción}
En la era digital actual, los sistemas de búsqueda juegan un papel crucial al ofrecer a los usuarios contenido y recomendaciones adaptadas a sus preferencias individuales. Sin embargo, a pesar de su precisión, estos sistemas carecen de transparencia, lo que genera desconfianza y falta de control por parte de los usuarios. Este informe aborda la necesidad de desarrollar un motor de búsqueda que permita a los usuarios consultas personalizadas y una trazabilidad detallada para entender cómo se toman las decisiones; y selecciones personalizadas para incidir en la necesidad de transparencia en este tipo de algoritmos.

\section*{¿Qué es la transparencia de los algoritmos?}
La transparencia de los algoritmos se refiere a la capacidad de comprender y analizar los mecanismos de toma de decisiones de los sistemas algorítmicos. Esto implica a menudo poder examinar el código fuente, los datos operativos y los criterios de decisión de los algoritmos.

Hay varias razones por las que los algoritmos deben ser transparentes. En primer lugar, hay una búsqueda de justicia e imparcialidad: las decisiones basadas en datos erróneos o sesgados pueden perjudicar a personas o grupos concretos.

Eli Pariser (2017) ha denominado la “burbuja de filtros” al efecto por el que un algoritmo selecciona las informaciones que prefiere el usuario, en función de las búsquedas anteriores, la geolocalización, o las preferencias en cualquier tipo de selecciones. La burbuja presenta una web empequeñecida con los mismos contenidos que coinciden con los propios puntos de vista, reforzando una ideología previa y al margen de un pensamiento crítico.

En el procesamiento de grandes datos se dan estereotipos muy significativos en los que basa su aprendizaje la IA. Aún lo es más si se tiene en cuenta que la ideología de las personas que deciden el problema a resolver con el algoritmo, sus objetivos y requisitos, también genera sesgos en el propio proceso de formulación.

Epstein y Robertson (2015) defienden que Google tiene posibilidad de cambiar el resultado del 25 \% de las elecciones nacionales de diferentes países del mundo sin que nadie advierta si se produce una manipulación en los resultados.

Así llegaron a la definición del Efecto Manipulador del Motor de Búsqueda o Search Engine Manipulation Effect (SEME) (Epstein y Robertson, 2015) sobre el que aseguran que es uno de los mayores efectos sobre el comportamiento humano que se haya sistematizado.

\section*{Descripción del tema}
Se hace necesario un motor de búsqueda que brinde la transparencia necesaria para evitar caer en el sesgo generado por los algoritmos actuales y por la infraestructura de la red. Por ello se presenta la opción de “Findtastic” que utiliza herramientas similares a las de los algoritmos predominantes, pero con la característica de permitir al usuario la personalización total y completa de estas herramientas en cada una de sus búsquedas.

\section*{Antecedentes de la temática seleccionada}
Otras iniciativas tratan de atajar el problema de los sesgos. El proyecto europeo FA*IR, con participación de la Universidad Pompeu Fabra y Eurecat, ha investigado la discriminación según género, procedencia o apariencia que lleva a invisibilizar a determinados colectivos en la búsqueda de empleo. El equipo de investigación ya ha desarrollado un algoritmo de búsqueda que invalida el sesgo étnico, de género o de edad, sin afectar al ranking obtenido. 

Otras alternativas se han centrado en indicadores que cuantifican y definen los sesgos, directos o indirectos, cuando se insertan en el software de un buscador o en otras utilidades basadas en el Big Data, con un modelo en la práctica (Bolukbasi et al., 2016) que se puede utilizar sin modificar las propiedades útiles del sistema. 

\end{document}